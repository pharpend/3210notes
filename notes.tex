\documentclass[12pt,letterpaper,oneside]{memoir}

\usepackage[backend=bibtex,date=iso8601]{biblatex}
\usepackage{color}
\usepackage{datetime2}
\usepackage{listings}
\usepackage{setspace}

\usepackage{hyperref}
\usepackage{cleveref}

\lstset{ %
  % backgroundcolor=\color{white},   % choose the background color; you must add \usepackage{color} or \usepackage{xcolor}
  basicstyle=\footnotesize\ttfamily,     % the size of the fonts that are used for the code
  breakatwhitespace=false,         % sets if automatic breaks should only happen at whitespace
  breaklines=false,                 % sets automatic line breaking
  captionpos=b,                    % sets the caption-position to bottom
  commentstyle=\color{mygreen},    % comment style
  deletekeywords={...},            % if you want to delete keywords from the given language
  escapeinside={\%*}{*)},          % if you want to add LaTeX within your code
  extendedchars=true,              % lets you use non-ASCII characters; for 8-bits encodings only, does not work with UTF-8
  frame=single,                    % adds a frame around the code
  keepspaces=true,                 % keeps spaces in text, useful for keeping indentation of code (possibly needs columns=flexible)
  % keywordstyle=\color{blue},       % keyword style
  % Actually, we are using Idris, but Haskell is close enough
  % language=\null,                % the language of the code
  % morekeywords={*,...},            % if you want to add more keywords to the set
  numbers=left,                    % where to put the line-numbers; possible values are (none, left, right)
  numbersep=5pt,                   % how far the line-numbers are from the code
  numberstyle=\tiny\ttfamily,    % the style that is used for the line-numbers
  % rulecolor=\color{mygray},        % if not set, the frame-color may be changed on line-breaks within not-black text (e.g. comments (green here))
  showspaces=false,                % show spaces everywhere adding particular underscores; it overrides 'showstringspaces'
  showstringspaces=false,          % underline spaces within strings only
  showtabs=false,                  % show tabs within strings adding particular underscores
  stepnumber=1,                    % the step between two line-numbers. If it's 1, each line will be numbered
  % stringstyle=\color{mymauve},     % string literal style
  tabsize=2,                       % sets default tabsize to 2 spaces
  % title=\lstname,                   % show the filename of files included with \lstinputlisting; also try caption instead of title
  % caption=\lstname ,                  % show the filename of files included with \lstinputlisting; also try caption instead of title
}

\setlength{\parskip}{1ex}
\addbibresource{notes.bib}

\begin{document}
\title{Math 3210 Notes}
\author{Peter Harpending}
\maketitle

\tableofcontents


\setcounter{chapter}{-1}
\chapter{Introduction}

This document contains my personal notes and elongated explanations of the
course material in Math 3210 taught at the University of Utah in Autumn 2015.

The official textbook is \textbf{\fullcite{taylor}}.

You can get the source for this document via git using \texttt{git clone
  git://github.com/pharpend/3210notes.git}.

This document is licensed under the Creative Commons Attribution-ShareAlike 4.0
International License. Said license can be found at
\url{https://creativecommons.org/licenses/by-sa/4.0/}.


\setcounter{chapter}{2}
\chapter{Continuous functions}

This chapter governs continuous functions.

\section{Continuity}

Continuous functions are informally described as ``functions whose graph can be
drawn without lifting up one's writing utensil''.

The func

\begin{appendices}
\end{appendices}

\begin{appendices}
    \chapter{Source code for generating graphs}
\label{apA-graphs}

The source code in this appendix is licensed under the BSD-2 license, found here

\lstinputlisting[label={bsd-license},caption={BSD License, for source code}]{graphs/BSDLICENSE}

\end{appendices}

\printbibliography

\end{document}