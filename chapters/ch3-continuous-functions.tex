\chapter{Continuous functions}

This chapter governs continuous functions.

\section{Continuity}

\subsection{Intuition and Examples}

Continuous functions are informally described as ``functions whose graph can be
drawn without lifting up one's writing utensil''. Here's a couple of examples,
to give you the intuition.

\answergraph{graphs/continuous-vs-discontinuous.png}

In this example, $f$ is continuous, while $g$ has a singularity at $x =
0$. Evaluating $g(0) = \frac{1}{0}$ is not possible. If you approach $0$ from
the left, you can see that $g$ decreases rather rapidly, to $-\infty$. If you
approach $0$ from the right, you see the opposite, where $g$ increases rapidly
to $\infty$, but never reaches a value at $x = 0$.

You should already be familiar with the basic notion of continuity, but that
graph might serve as a reminder.

Note that a function can be defined at every point, but still be
discontinuous. Here's another silly instance of discontinuity.

\answergraph{graphs/constant-discontinuity.png}

The function $h$ is defined at every point in $\R$ ($g$ is not). However, $h$
does have a discontinuity at $x = 0$.

Continuity is usually qualified with a domain. The function $f$ from earlier is
continuous over all of $\R$. $g$, the reciprocal function, is continuous on
every real number except $0$, or

\begin{equation}
    \label{eq:r-minus-0}
    \R \setminus \mset{0}
\end{equation}

\Cref{eq:r-minus-0} could be written in interval notation, as seen in
\cref{eq:r-sans-0-interval-notation}.

\begin{equation}
    \label{eq:r-sans-0-interval-notation}
    \label{eq:r-sans-0}
    \mlist{-\infty, 0} \cup \mlist{0, \infty}
\end{equation}

The function $h$, the bi-constant function with a trivial discontinuity at $x =
0$, is continuous on $\mlist{\infty,0}$ and on $\mclop{0, 1}$. It would be a
mistake, however, to say that $h$ is continuous on

\begin{equation}
    \mlist{\infty, 0} \cup \mclop{0, 1}
\end{equation}

because 

\begin{equation}
    \mlist{\infty, 0} \cup \mclop{0, 1} = \R
\end{equation}

and $h$ is not continuous over all of $\R$.

Note that \cref{eq:r-sans-0} does not have this problem, because

\begin{equation}
    0 \notin \mlist{-\infty, 0} \cup \mlist{0, \infty}
\end{equation}

\subsection{The formal definition}

\begin{appendices}
\end{appendices}